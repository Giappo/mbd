\documentclass[]{article}
\usepackage{lmodern}
\usepackage{amssymb,amsmath}
\usepackage{ifxetex,ifluatex}
\usepackage{fixltx2e} % provides \textsubscript
\ifnum 0\ifxetex 1\fi\ifluatex 1\fi=0 % if pdftex
  \usepackage[T1]{fontenc}
  \usepackage[utf8]{inputenc}
\else % if luatex or xelatex
  \ifxetex
    \usepackage{mathspec}
  \else
    \usepackage{fontspec}
  \fi
  \defaultfontfeatures{Ligatures=TeX,Scale=MatchLowercase}
\fi
% use upquote if available, for straight quotes in verbatim environments
\IfFileExists{upquote.sty}{\usepackage{upquote}}{}
% use microtype if available
\IfFileExists{microtype.sty}{%
\usepackage{microtype}
\UseMicrotypeSet[protrusion]{basicmath} % disable protrusion for tt fonts
}{}
\usepackage[margin=1in]{geometry}
\usepackage{hyperref}
\hypersetup{unicode=true,
            pdftitle={Mbd derivation},
            pdfauthor={Giovanni Laudanno},
            pdfborder={0 0 0},
            breaklinks=true}
\urlstyle{same}  % don't use monospace font for urls
\usepackage{color}
\usepackage{fancyvrb}
\newcommand{\VerbBar}{|}
\newcommand{\VERB}{\Verb[commandchars=\\\{\}]}
\DefineVerbatimEnvironment{Highlighting}{Verbatim}{commandchars=\\\{\}}
% Add ',fontsize=\small' for more characters per line
\usepackage{framed}
\definecolor{shadecolor}{RGB}{248,248,248}
\newenvironment{Shaded}{\begin{snugshade}}{\end{snugshade}}
\newcommand{\AlertTok}[1]{\textcolor[rgb]{0.94,0.16,0.16}{#1}}
\newcommand{\AnnotationTok}[1]{\textcolor[rgb]{0.56,0.35,0.01}{\textbf{\textit{#1}}}}
\newcommand{\AttributeTok}[1]{\textcolor[rgb]{0.77,0.63,0.00}{#1}}
\newcommand{\BaseNTok}[1]{\textcolor[rgb]{0.00,0.00,0.81}{#1}}
\newcommand{\BuiltInTok}[1]{#1}
\newcommand{\CharTok}[1]{\textcolor[rgb]{0.31,0.60,0.02}{#1}}
\newcommand{\CommentTok}[1]{\textcolor[rgb]{0.56,0.35,0.01}{\textit{#1}}}
\newcommand{\CommentVarTok}[1]{\textcolor[rgb]{0.56,0.35,0.01}{\textbf{\textit{#1}}}}
\newcommand{\ConstantTok}[1]{\textcolor[rgb]{0.00,0.00,0.00}{#1}}
\newcommand{\ControlFlowTok}[1]{\textcolor[rgb]{0.13,0.29,0.53}{\textbf{#1}}}
\newcommand{\DataTypeTok}[1]{\textcolor[rgb]{0.13,0.29,0.53}{#1}}
\newcommand{\DecValTok}[1]{\textcolor[rgb]{0.00,0.00,0.81}{#1}}
\newcommand{\DocumentationTok}[1]{\textcolor[rgb]{0.56,0.35,0.01}{\textbf{\textit{#1}}}}
\newcommand{\ErrorTok}[1]{\textcolor[rgb]{0.64,0.00,0.00}{\textbf{#1}}}
\newcommand{\ExtensionTok}[1]{#1}
\newcommand{\FloatTok}[1]{\textcolor[rgb]{0.00,0.00,0.81}{#1}}
\newcommand{\FunctionTok}[1]{\textcolor[rgb]{0.00,0.00,0.00}{#1}}
\newcommand{\ImportTok}[1]{#1}
\newcommand{\InformationTok}[1]{\textcolor[rgb]{0.56,0.35,0.01}{\textbf{\textit{#1}}}}
\newcommand{\KeywordTok}[1]{\textcolor[rgb]{0.13,0.29,0.53}{\textbf{#1}}}
\newcommand{\NormalTok}[1]{#1}
\newcommand{\OperatorTok}[1]{\textcolor[rgb]{0.81,0.36,0.00}{\textbf{#1}}}
\newcommand{\OtherTok}[1]{\textcolor[rgb]{0.56,0.35,0.01}{#1}}
\newcommand{\PreprocessorTok}[1]{\textcolor[rgb]{0.56,0.35,0.01}{\textit{#1}}}
\newcommand{\RegionMarkerTok}[1]{#1}
\newcommand{\SpecialCharTok}[1]{\textcolor[rgb]{0.00,0.00,0.00}{#1}}
\newcommand{\SpecialStringTok}[1]{\textcolor[rgb]{0.31,0.60,0.02}{#1}}
\newcommand{\StringTok}[1]{\textcolor[rgb]{0.31,0.60,0.02}{#1}}
\newcommand{\VariableTok}[1]{\textcolor[rgb]{0.00,0.00,0.00}{#1}}
\newcommand{\VerbatimStringTok}[1]{\textcolor[rgb]{0.31,0.60,0.02}{#1}}
\newcommand{\WarningTok}[1]{\textcolor[rgb]{0.56,0.35,0.01}{\textbf{\textit{#1}}}}
\usepackage{longtable,booktabs}
\usepackage{graphicx,grffile}
\makeatletter
\def\maxwidth{\ifdim\Gin@nat@width>\linewidth\linewidth\else\Gin@nat@width\fi}
\def\maxheight{\ifdim\Gin@nat@height>\textheight\textheight\else\Gin@nat@height\fi}
\makeatother
% Scale images if necessary, so that they will not overflow the page
% margins by default, and it is still possible to overwrite the defaults
% using explicit options in \includegraphics[width, height, ...]{}
\setkeys{Gin}{width=\maxwidth,height=\maxheight,keepaspectratio}
\IfFileExists{parskip.sty}{%
\usepackage{parskip}
}{% else
\setlength{\parindent}{0pt}
\setlength{\parskip}{6pt plus 2pt minus 1pt}
}
\setlength{\emergencystretch}{3em}  % prevent overfull lines
\providecommand{\tightlist}{%
  \setlength{\itemsep}{0pt}\setlength{\parskip}{0pt}}
\setcounter{secnumdepth}{5}
% Redefines (sub)paragraphs to behave more like sections
\ifx\paragraph\undefined\else
\let\oldparagraph\paragraph
\renewcommand{\paragraph}[1]{\oldparagraph{#1}\mbox{}}
\fi
\ifx\subparagraph\undefined\else
\let\oldsubparagraph\subparagraph
\renewcommand{\subparagraph}[1]{\oldsubparagraph{#1}\mbox{}}
\fi

%%% Use protect on footnotes to avoid problems with footnotes in titles
\let\rmarkdownfootnote\footnote%
\def\footnote{\protect\rmarkdownfootnote}

%%% Change title format to be more compact
\usepackage{titling}

% Create subtitle command for use in maketitle
\providecommand{\subtitle}[1]{
  \posttitle{
    \begin{center}\large#1\end{center}
    }
}

\setlength{\droptitle}{-2em}

  \title{Mbd derivation}
    \pretitle{\vspace{\droptitle}\centering\huge}
  \posttitle{\par}
    \author{Giovanni Laudanno}
    \preauthor{\centering\large\emph}
  \postauthor{\par}
      \predate{\centering\large\emph}
  \postdate{\par}
    \date{2019-07-24}

\usepackage{bbm}

\begin{document}
\maketitle

{
\setcounter{tocdepth}{2}
\tableofcontents
}
\hypertarget{the-theory}{%
\section{The theory}\label{the-theory}}

\hypertarget{the-process}{%
\subsection{The process}\label{the-process}}

Mbd stands for Multiple-Birth-Death. It is a model that includes the possibility
of the occurrance of multiple simultaneous speciations at any given time.
The process allows 3 possible events to occurr:

\begin{itemize}
\tightlist
\item
  single speciation, with rate \(\lambda\);
\item
  extinction, with rate \(\mu\);
\item
  multiple speciation, with rate \(\nu\). If it occurs each of the species present has probability \(q\) to speciate;
\end{itemize}

Parameters \(\lambda\) and \(\mu\) reproduce the effect of the standard birth-death
model. The novelty comes from the possibility of multiple speciation.
However all the species present at a given time are not all equal. Part of them
are in fact visible in the reconstructed tree and part are not, because they will go extinct before the
present. We will label these as \(k\)- and \(m\)-species, respectively.
Having two different pools, we will consider separately the speciations coming
from each pool, according to the simple binomial law mentioned before.

\hypertarget{transition-from-the-m-pool}{%
\subsubsection{\texorpdfstring{Transition from the \(m\)-pool}{Transition from the m-pool}}\label{transition-from-the-m-pool}}

The \(m\)-pool can only produce new species in the \(m\)-pool, as if visible species
were produced this will contradict the definition itself of invisible species.

The probability \(M\) of multiple speciation from the \(m\)-pool leading to the change
of state \((m,k) \rightarrow (m + i, k)\) is

\[
M^{k,k}_{m,m + i} = \binom{m}{i} q ^ i (1 - q) ^ {m - i}
\]

\hypertarget{transition-from-the-k-pool}{%
\subsubsection{\texorpdfstring{Transition from the \(k\)-pool}{Transition from the k-pool}}\label{transition-from-the-k-pool}}

If the multiple speciation occurs starting from the \(k\)-pool things are more complex,
as this can change the number of species in both pools.

The probability \(K\) of multiple speciation from the \(k\)-pool leading to the change
of state \((m,k) \rightarrow (m + i, k + b)\) is

\[
K^{k, k + b}_{m, m + j} = 2 ^ j \binom{k}{b} \binom{k - b}{j} q ^ {b + j} (1 - q) ^ {k - b - j}
\]

Here the factor \(2 ^ j\) comes from the fact that for each of the \(j\) speciations
from the \(k\)-pool to the \(m\)-pool has two different ways to realize. In fact,
having two species, say 1 and 2, in the process of reconstructing
the tree we could interpret 1 as the visible and 2 as the invisible or the other
way around. Having \(j\) new species the number of possibilities becomes \(2 ^ j\).
This factor arises for the same reason of the factor \(2k\) in the lambda term
of the original Q-equation in Etienne et al. (2011) (see also Q-equation below).

\hypertarget{full-transition-from-both-pools}{%
\subsubsection{Full Transition from both pools}\label{full-transition-from-both-pools}}

If we want to consider the full process we have to combine the two probabilities
\(K^{k, k + b}_{m, m + j}\) and \(M^{k,k}_{m, m + i}\).

The probability \(P\) of multiple speciation from both pools leading to the change
of state \((m,k) \rightarrow (m + i + j, k + b)\) is the convolution of the two
terms:

\[
\begin{aligned}
N^{k, k + b}_{m, m + a} & = [K^{k, k + b}_{m, m + j} * M^{k,k}_{m, m + i}]^{k, k + b}_{m, m + a} \\
& = \binom{k}{b} q ^ b (1 - q) ^ {k + m - b}
\sum_j 2 ^ j \binom{k - b}{j} \binom{m - a} {a - j} q^a (1 - q) ^{-2a}
(\#eq:ndefinition)
\end{aligned}
\]

where \(a = i + j\).

\hypertarget{the-likelihood}{%
\subsection{The likelihood}\label{the-likelihood}}

Mbd is a likelihood-based model. The structure of the likelihood is taken from
Etienne et al. (2011). Such framework is built around the core function
\(Q_{m}^{k}(t)\) which represents the probability to have, at the time \(t\), \(k\)
species visible in the phylogeny as well as \(m\) additional species that will
go extinct before the present time. In the following we will refer to visible
species always with the letter \(k\) and to unseen species with indexes \(m\)
or \(n\).
In this context the likelihood is defined as

\[
L = \frac{Q_{m = 0}^{k = k_{p}}(t_{p})}{P_c}
\]

where \(t_p\) is the present time and \(k_p\) is the number of visible species at
the present, namely the number of tips. \(P_c\) stands for a possible conditioning
probability.
To obtain the vector \(\mathbf{Q}^k(t)\) at the present one must integrate it from
the crown (or stem) age to the present. The starting vector at the base of the
tree is

\[
Q_{m}^{k_{1}}(t_{1}) = \delta_{m,0}
\]

Given the vector of the phylogeny's branching times
\(\mathbf{t} = (t_1, \dots, t_p)\), the integration scheme can be
summarized by the following

\[
 \mathbf{Q}^{k_p}(t_p) = A^{k_p}(t_{k_p-1},t_p)\, B^{k_p-1,k_p}\,
    A^{k_p-1}(t_{k_p-2},t_{k_p-1}) \ldots \notag \\
 \qquad A^{4}(t_3,t_4)\,B^{3,4}\,A^{3}(t_2,t_3)\,
    B^{2,3}\, A^{2}(t_1,t_2)\, \mathbf{Q}^2(t_1)
\]

where \(B^{k - 1, k}\) and \(A^k\) are matrices that take into account how
the \(\mathbf{Q}^k\) has to be modified, respectively, on branching times and in
the time intervals between consecutive branching times. Note that on the
elements \(A\), \(B\) and \(\mathbf{Q}\) the subscript is not to be intended as a
power, but it is just an index to keep track of the k-species.

\hypertarget{a-operator}{%
\subsubsection{A operator}\label{a-operator}}

The A operator is given by the integration of a set of differential equations
between two consecutive nodes. So, defined the set in the time interval
\([t_{i-1}, t_i]\), where k species are present in the phylogeny, as:

\[
\frac{d}{dt}Q^k_m(t) = \Sigma_n T^{k,k}_{m,n} \cdot Q^k_n(t)
\]

where m, n, label the amount of unseen species in the phylogeny,
A is thus defined as:

\[
A(t_i - t_{i-1}) = e^{T^{k,k}(t_k - t_{k-1})}
\]

And the formal solution in \(t = t_{i}\) from initial conditions at \(t_{i - 1}\) is

\[
Q^k_m(t_{i}) = A_{m,n}(t_i - t_{i-1}) Q^k_n(t_{i - 1})
\]

The mbd package allows to perform this in R with the function:

\begin{Shaded}
\begin{Highlighting}[]
\NormalTok{a <-}\StringTok{ }\NormalTok{mbd}\OperatorTok{:::}\KeywordTok{a_operator}\NormalTok{(}
  \DataTypeTok{q_vector =} \KeywordTok{c}\NormalTok{(}\DecValTok{1}\NormalTok{),}
  \DataTypeTok{transition_matrix =} \KeywordTok{c}\NormalTok{(}\DecValTok{1}\NormalTok{),}
  \DataTypeTok{time_interval =} \FloatTok{0.1}
\NormalTok{)}
\CommentTok{#> Registered S3 method overwritten by 'R.oo':}
\CommentTok{#>   method        from       }
\CommentTok{#>   throw.default R.methodsS3}
\end{Highlighting}
\end{Shaded}

\hypertarget{transition-matrix-and-q-equation}{%
\subsubsection{Transition matrix and Q equation}\label{transition-matrix-and-q-equation}}

The matrix \(T^k\) summarizes the system of ordinary differential equations (ODE)
used to evolve in time the vector \(\mathbf{Q}^k\) in a given time interval with
a fixed amount of k-species. The ODE system is the following

\[
\begin{aligned}
\frac{d Q^k_m(t)}{dt} & = 
\lambda (m + 2k - 1) Q^k_{m - 1}(t) +
\mu (m + 1) Q^k_{m + 1}(t) \\
& + \nu (1 - q) ^ {k + m} 
\sum_{a}
\sum_{j} 2 ^ {j} \binom{k}{j} \binom{m - a}{a - j} q ^ {a} (1 - q) ^ {- 2a}
Q^k_{m - a}(t) \\
& - (\lambda + \mu) (m + k) Q^k_{m}(t)
- \nu (1 - (1 - q) ^ {m + k}) Q^k_{m}(t)
\end{aligned}
\]
Here the two \(\nu\) terms are obtained following the reasoning explained in the
previous section, when the multiple speciation mechanism has been described.

\hypertarget{q-equation-matrix-form}{%
\subsubsection{Q equation: matrix form}\label{q-equation-matrix-form}}

The way the ODE set is implemented in the mbd package is by using matrices.
It is possible to decompose the transition matrix \(T^k\) in the following way

\[
T^{k,k}_{m,n} = \lambda L^{k,k}_{m,n} + \mu D^{k,k}_{m,n} + \nu N^{k,k}_{m,n}
\]
The matrices \(L^{k,k}\) and \(D^{k,k}\) are defined according to the standard birth-death
model, so

\[
\begin{aligned}
L^{k,k}_{m,n} & = \delta_{m, n + 1} (m + 2k - 1) \\
D^{k,k}_{m,n} & = \delta_{m, n - 1} (m + 1)
\end{aligned}
\]

The matrix \(N^{k,k}\) instead is given by @ref(eq:ndefinition) for components
\(N^{k,k}_{m,n}\) with \(m > n\) and by \(1 - (1 - q) ^ {m + k}\) for \(m = n\).
This can be stated explicitly using the relation \(m = n + i + j = n + a\):

\[
\begin{aligned}
N^{k,k}_{m,n} & =
(1 - (1 - q) ^ {k + m}) 
\qquad & \text{if} \qquad
m = n
\\
N^{k,k}_{m,n} & =
(1 - q) ^ {k + m}
\sum_j 2 ^ j \binom{k}{j} \binom{m - a}{a - j} q ^ a (1 - q) ^ {-2a}
\qquad & \text{if} \qquad
m > n
\end{aligned}
\]

\hypertarget{b-operator}{%
\subsection{B operator}\label{b-operator}}

\hypertarget{section}{%
\subsubsection*{}\label{section}}
\addcontentsline{toc}{subsubsection}{}

\hypertarget{refs}{}
\leavevmode\hypertarget{ref-etienne2012diversity}{}%
Etienne, Rampal S, Bart Haegeman, Tanja Stadler, Tracy Aze, Paul N Pearson, Andy Purvis, and Albert B Phillimore. 2011. ``Diversity-Dependence Brings Molecular Phylogenies Closer to Agreement with the Fossil Record.'' \emph{Proceedings of the Royal Society B: Biological Sciences} 279 (1732): 1300--1309.


\end{document}
